% upLaTeX用にuplatexオプションが指定してあるが、
% pLaTexの場合ははずす
\documentclass[pdflatex, ja=standard, a4paper]{bxjsarticle}

\usepackage{amsmath, amssymb, amsthm}
\usepackage{ascmac}
\usepackage{graphicx}
\usepackage{ascmac}
\usepackage{enumitem}

\DeclareGraphicsRule{.ai}{pdf}{.ai}{}

\newcommand{\source}{\mathcal{S}}
\newcommand{\code}{\mathcal{C}}

\newcommand{\real}{\mathbb{R}}
\newcommand{\aff}{\mathop{\mathrm{aff}}}
\newcommand{\ri}{\mathop{\mathrm{relint}}}
\newcommand{\cl}{\mathop{\mathrm{cl}}}

\begin{document}
\begin{description}[style=nextline]
    \item[$x_1$と$x_2$を通る直線] $\{\theta x_1 + (1 - \theta) x_2 : \theta \in \real\}$。$\theta x_1 + (1 - \theta) x_2 = \theta (x_1 - x_2) + x_2$から、$x_2$を起点として、$x_1 - x_2$を$\theta$倍したものを足せば、その直線上の任意の点を表すことができる
    \item[$x_1$と$x_2$を結ぶ線分] $\{\theta x_1 + (1 - \theta) x_2 : \theta \in [0, 1]\}$。直線において、$\theta$の動く範囲を$[0, 1]$に制限すれば、$x_1$と$x_2$を結ぶ線分上の点を表すことができる
    \item[基点] $\theta (x_1 - x_2) + x_2$における$x_2$
    \item[方向] $\theta (x_1 - x_2) + x_2$における$x_1 - x_2$
    \item[アフィン空間] 集合内の任意の異なる2点を通る直線を全て含む集合
    \item[点$x_1, \ldots, x_k$のアフィン結合] 任意の$\theta_1 + \cdots + \theta_k = 1$となる$\theta_1, \ldots, \theta_k$を用いて、$\theta_1 x_1 + \cdots + \theta_k x_k$と表される点。$x_1$と$x_2$の直線上の点$\theta x_1 + (1 - \theta) x_2$という式を一般に$k > 2$に対しても拡張した形
    \item[アフィン空間の次元] アフィン空間内の任意の点から、同じアフィン空間内のある点$x_0$を引いた集合は、$x_0 - x_0 = 0$とアフィン空間が任意の空間内の異なる2点間の直線を含んでいることから線形空間となる。この線形空間の次元のことをいう
    \item[集合$C$のアフィン包$\aff C$] $\{\theta_1 x_1 + \cdots + \theta_k x_k : \theta_1 + \cdots + \theta_k = 1, x_1, \ldots, x_k \in C\}$。集合$C$内の任意の点同士のアフィン結合を集めた集合。$C$を包む最小のアフィン集合でもある
    \item[中心が$x$、半径$r$の球$B(x, r)$] $\{y : \|y - x\| \leq r\}$。距離が$r$以下となる点を集めた集合。たとえば、3次元空間内では球、2次元空間内では円
    \item[集合$C$の相対的内部$\ri C$] $\{x \in C : \exists r > 0; B(x, r) \cap \aff C \subseteq  C\}$。$C$の内部を表現したもの。アフィン包$\aff C$との共通部分をとっているのは、アフィン空間の次元が、全体の空間よりも低い場合があるためである。例えば、3次元空間内の平面上の円の内部を表すには、球が存在するかだけではいえない。なぜならば、3次元空間内の球はその定義から、平面から必ず飛び出す点をもってしまう。そこで、アフィン包との共通部分をとることで、平面内の点だけで議論することができるようになる
    \item[集合$C$の相対的内点] $C$の相対的内部$\ri C$内の点
    \item[閉集合$F$] 集合$F$内の任意の点列$\{x_n\} \subseteq F$は収束して、その収束先はまた$F$に属するような集合のこと。球は閉集合となる
    \item[集合$C$の閉包$\cl C$] $C$を包む最小の閉集合
    \item[集合$C$の相対的境界] $\cl C - \ri C$。集合$C$の境界をさす
\end{description}
% \begin{figure}[b]
%     \centering
%     \begin{tabular}{c}
%         \begin{minipage}{0.5\hsize}
%             \centering
%             \includegraphics{image/code-and-source.ai}
%             \caption{情報源$\source$はシンボル列$s_{i_1} s_{i_2} \cdots$を出し、それを符号$\code$を利用して語の列$w_{i_1} w_{i_2} \cdots$へ符号化}
%         \end{minipage}
%         \begin{minipage}{0.5\hsize}
%             \centering
%             \includegraphics{image/tree.ai}
%             \caption{高さが$2$の$2$元根付き木の例}
%         \end{minipage}
%     \end{tabular}
% \end{figure}
\end{document}
